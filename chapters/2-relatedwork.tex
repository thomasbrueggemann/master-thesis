\section{Combining Source Code Analysis with Information Privacy Risk Assessment}

\mH apps have been examined in various research studies that aim at providing insights for developers as well as users into how private information is processed.
Privacy issues are the most impactful user complaint while using mobile apps.\footnote{See \cite{Khalid2015}, p. 5.}
This encourages research to address information privacy risks.

Research focus has been put on the technical side of information privacy breach. 
It has been analyzed, to what degree the data storage in internal Android log files or on the memory card within a phone or tablet poses a threat to users information privacy.\footnote{For the previous two sentences, see \cite{He2014}, p. 645-646.}
Technical evaluation of mobile apps even goes further into the topics of decompilation to analyze device identification or geolocation data leaks.\footnote{See \cite{Mcclurg2012}, p. 1, 5., \cite{Enck2011}, p. 1. and \cite{Mitchell2013}, p.6-7.}
Decompilation reveals to be a feasible assessment technique for \ipr and data leaks.

In informatics and software development contexts, \sca has been used to analyze source code and provide feedback on coding styles to the users  while programming or "to find defects in programs"\footnote{\cite{Bardas2010}, p. 1.}.
\Sca provides a fast way to analyze source code\footnote{See \cite{Bardas2010}, p. 5.}, which makes it suitable to automate the assessment of large datasets.
A further benefit of using \sca to retrieve information from software is that the software does not need to be executed during the analyzation process.
This additionally supports the development of fast performing assessment tools that are suitable for application on large datasets of source code since there is no need to wait for the application runtime to execute the software.

Our study will use the benefits of \sca and apply them to the assessment of \mH information privacy risks.
It is unclear if \sca is a viable tool to analyze and identify information privacy risk factors.
We will use the comprehensive privacy-risk-relevant information privacy practices that \cite{Dehling2016} identified\footnote{See \cite{Dehling2016}, p. 8-17.} and try to implement \sca strategies to identify those risks automatically.
This will be a vital addition to current research, since there is yet no holistic approach to apply \sca to \ipr detection that takes an ample amount of \iprfs into account.

\subsection{Information Privacy Risk Assessment}

\subsection{Static Code Analysis}

\subsection{Relevant Information Privacy Risk Factors}