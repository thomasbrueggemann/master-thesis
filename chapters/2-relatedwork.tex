\section{Combining Source Code Analysis with Information Privacy Risk Assessment}

\mH apps have been examined in various research studies that aim at providing insights for developers as well as users into how private information is processed.
Privacy issues are the most impactful user complaint while using mobile apps.\footnote{See \cite{Khalid2015}, p. 5.}
This encourages research to address information privacy risks.

Research focus has been put on the technical side of information privacy breach. 
It has been analyzed, to what degree the data storage in internal Android log files or on the memory card within a phone or tablet poses a threat to users information privacy.\footnote{For the previous two sentences, see \cite{He2014}, p. 645-646.}
Technical evaluation of mobile apps even goes further into the topics of decompilation to analyze device identification or geolocation data leaks.\footnote{See \cite{Mcclurg2012}, p. 1, 5., \cite{Enck2011}, p. 1. and \cite{Mitchell2013}, p.6-7.}
Decompilation reveals to be a feasible assessment technique for \ipr and data leaks.

In informatics and software development contexts, \sca has been used to analyze source code and provide feedback on coding styles to the users  while programming or "to find defects in programs"\footnote{\cite{Bardas2010}, p. 1.}.
\Sca provides a fast way to analyze source code\footnote{See \cite{Bardas2010}, p. 5.}, which makes it suitable to automate the assessment of large datasets.
A further benefit of using \sca to retrieve information from software is that the software does not need to be executed during the analyzation process.
This additionally supports the development of fast performing assessment tools that are suitable for application on large datasets of source code since there is no need to wait for the application runtime to execute the software.

Our study will use the benefits of \sca and apply them to the assessment of \mH information privacy risks.
It is unclear if \sca is a viable tool to analyze and identify information privacy risk factors.
We will use the comprehensive privacy-risk-relevant information privacy practices that \cite{Dehling2016} identified\footnote{See \cite{Dehling2016}, p. 8-17.} and try to implement \sca strategies to identify those risks automatically.
This will be a vital addition to current research, since there is yet no holistic approach to apply \sca to \ipr detection that takes an ample amount of \iprfs into account.

\subsection{Information Privacy Risk Assessment}

\subsection{Static Code Analysis}

\subsection{Relevant Information Privacy Risk Factors}

\todo{Wenn es etwas ist über das ich den Nutzer irgendwie informieren sollte, dann ist es ja auch in irgendeiner Form ein Privacy Risiko}
\todo{Auch Fälle identifizieren, die man vielleicht noch nicht verstanden hat!}
\todo{Nicht nur was anderes mit SCA umsetzen, sondern was kann ich weiter damit machen.}

For this thesis, we will use the set of \ipp extracted from literature, the Platform for Privacy Preferences (\acs{P3P}) guide\footnote{https://www.w3.org/TR/P3P/, visited 06/06/2016} and app reviews by \cite{Dehling2016} as a source to derive \iprfs from. \footnote{For this and the following sentence, see \cite{Dehling2016}, p. 1-2.} 
\Ipp are common practices of informing users about the information privacy practices of an app that app providers should follow in order to achieve higher levels of transparency.
A hierarchy of \ipp is formed by clustering the \ipp by their content aspects. 
The top level of the hierarchy are 'Content' and 'Practice'.
The 'Content' hierarchy level contains sub-hierarchy branches that express \ipp concerning the handling of information content, the information collection content and meta content about information collection, information on offered privacy controls and information on what purpose the \ipp was collected for.
The sub-hierarchy branches of the top-level 'Practice' contain e.g. information on the existence of dispute resolution practices or access rights practices of the users.

We argue that if a \ipp is a circumstance that the user should be informed about, an \ipp expresses an \ipr to the app user.
But since not all of the enlisted \ipp express or imply an \ipr to app users, we review and extract the \ipp that are relevant in terms of posing and expressing a potential information privacy risk.
We will further limit the \ipp by excluding \ipp that are known to be technically infeasible to detect via static code analysis.
An example for such an exclusion is the \ipp 'InformationRetentionContent', which captures, if an app provider carries out a certain information retention policy or not.\footnote{See \cite{Dehling2016}, p. 8.}
This is a feature that is undetectable by \sca and beyond the scope of app source code analysis.
An analysis of the app providers backend system would be necessary to ensure that the collection information is retained according to the app providers policy promise.

We include a full list of all \ipp in Appendix A including detailed comments on the technical limitations, if any, towards the \sca detection of each information privacy practice.

The following \ipp were identified as relevant to the \sca and further inspection within this thesis:

The complete hierarchy \textit{H2} 'InformationSecurityContent' can be analysed via \sca including the \ipp 'SecurityDuringProcessingContent', 'SecurityDuringStorageContent' and 'SecurityDuringTransferContent'.
Partially supported will be the hierarchy \textit{H3} 'InformationSharingContent'. Analysis will be applied to the containing \ipp 'SharingWithAdvertiserContent', 'SharingWithAggregatorContent', 'SharingWithAnalystContent', 'SharingWithDeliveryContent', 'SharingWithGovernmentContent', 'SharingWithOtherUsersContent', 'SharingWithProviderAgentsContent', 'SharingWithPublicContent' and 'SharingWithUserAuthorizedContent'.
