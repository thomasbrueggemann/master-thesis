\section{Feasibility of Automated Information Privacy Risk Assessment}

Within the following chapter we want to express the results of the implementation and evaluation phase and assess to what degree the automation of information privacy risk assessment is feasible.
We will also present the results on how well the \aiprat performs in comparison to human researchers.

\subsection{The Automated Information Privacy Risk Assessment of Free Android mHealth Apps}

\subsubsection{Download Phase}

The original dataset from the \cite{Xu2015} repository of app store listings contains 5379 app entries from the Google PlayStore in the category 'Medical' and 'Health and Fitness'.
From this original dataset we extracted the 3180 free apps for further inspection.
It was possible to download 2250 app APK files via the undocumented Google API.
The remaining 930 APK files either returned a Google authentication error or were not available on the Google PlayStore anymore.

Downloading the 2250 APK files took multiple download-runs over the whole dataset, since the Google API only allows a certain amount of download requests per time unit.
The number of allowed download requests varied throughout the download phase and could not be detected exactly.
Various tests downloading APK files automatically via websites like apkpure.com\footnote{https://apkpure.com/, visited 06/05/2016} or apk-dl.com\footnote{https://apk-dl.com, visited 06/05/2016} failed due to those websites effectively blocking automated non-browser traffic.
The download of the 2250 APK files took 11 days in total and 18 restarts of the download script.

\subsubsection{Decompilation Phase}

\subsubsection{Static Code Analysis Phase}

\todo{Live data flow analysis does not work because of computation resources}

\subsection{Evaluation of the Automated Information Privacy Risk Assessment Tool}