\addcontentsline{toc}{section}{Appendices}
\section*{Appendices}

\subsection*{Appendix A}

\textit{/appendices/Results.xlsx}. 

A list of the found/not-found results of information privacy risks per information privacy practice, detected by the coherent \AIPRAT strategies.
The table includes summerizations of the occurance of each strategy result and the total number of found strategy results per inspected app.

\subsection*{Appendix B}

\textit{/appendices/ResearcherReview.xlsx}

The information privacy practices that were found to be relevant as an information privacy risk listed for researchers to be reviewed.
Each app is reviewed independently and its source code inspected for the information privacy risk occurrences.
This is analogous to the automated information privacy risk assessment tool and provides space for a comparison of human versus computer in the case of information privacy risk assessment via static code analysis.
The researcher have to claim if they found an occurrence of a specific information privacy risk and at what certainty level (low, medium, high), depending on how certain they are about the found.
Lastly, the researchers have a chance to leave a comment on the information privacy risk found for enhanced understanding.